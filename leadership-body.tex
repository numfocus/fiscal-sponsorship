%% Each project should read this document in its entirety, decide what its
%% leadership model will be and write it up in detail here.  The content of
%% this file is meant only as a set of examples and guidance.

[\textbf{FIXME: Note:} This section should describe the way in which the
  Project wishes to interface with NumFOCUS; including who has authority to
  communicate with NumFOCUS regarding the Project and what is required in order
  for NumFOCUS to act on behalf of the Project. For example, all of the
  Contributors confirm their approval of a certain action, or can any one of
  the Contributors instruct NumFOCUS to take a certain action. Also, the
  Contributors may want to identify certain other individuals that have the
  power to represent the Project. Here a few samples of how projects have
  handled this clause in the other projects:

\begin{itemize}

\item \textbf{Simple Self-Perpetuating Committee}. The \signatories{}, each a
  signatory hereto, shall initially [FIXME: form or comprise] the
  \leadershipbody{} as a Project Committee (``Committee'') to represent the
  Project in its official communication with NumFOCUS.  Existing Project
  Committee Members (``Members'') can be removed from and new Members can be
  added to the Committee by simple majority vote of the existing Committee;
  however, three (3) shall be the mandatory minimum number of Members. All
  decisions of the Committee shall be made by simple majority. The Committee
  shall appoint, by majority vote, one Member as its ``Representative'' to
  communicate all Project decisions to NumFOCUS.


The Representative shall promptly inform NumFOCUS of changes in the Committee
composition and of contact information for all Members.  If NumFOCUS is unable,
after all reasonable efforts, to contact a majority of the Members for a period
of sixty (60) days, or if the minimum number of Members is fewer than three for
a period of at least sixty days, NumFOCUS may unilaterally appoint new Members
from the Project community to replace any unreachable Members and/or to
increase the Committee composition to three Members.

\item \textbf{Self-Perpetuating Committee, w/ avoiding employees of same
  company serving together}. The \signatories{}, each a signatory hereto, shall
  initially [FIXME: form or comprise] the \leadershipbody{} as a Project
  Committee (``Committee'') to represent the Project in its official
  communication with NumFOCUS. Existing Project Committee Members (``Members'')
  can be removed from and new Members can be added to the Committee by simple
  majority vote of the existing Committee; however, three (3) (the ``Minimum'')
  shall be the mandatory minimum number of Members.


For purposes of this Agreement, a Member is ``Employed'' by an Entity if the
Member is compensated for more than thirty-two (32) hours of work per week by
such Entity for a continuous period of more than sixty (60) days. No more than
one Member may be Employed by the same Entity.


Should two (2) or more Members be Employed by the same Entity at any time
(e.g., if an existing Member changes employers while a Member), Members
Employed by the same Entity must immediately resign in succession until only
one (1) of them remains on the Committee. Should voluntarily resignations fail
to yield the aforementioned result after sixty (60) days, the Members Sharing
an Employer shall be removed by NumFOCUS from the Committee in order of
decreasing seniority. Seniority shall be determined by length of service by the
Member on the Committee, including all historical periods of non-contiguous
service.


All decisions of the Committee shall be made by simple majority. The Committee
shall appoint, by majority vote, one Member as its Representative to
communicate all Project decisions to NumFOCUS. The Representative shall
promptly inform NumFOCUS of changes in the Committee composition and of contact
information for all Members. If NumFOCUS is unable, after all reasonable
efforts, to contact a majority of the Members for a period of sixty (60) days,
or if the number of Members is fewer than the Minimum for a period of at least
sixty days, NumFOCUS may, after at least thirty days notice to Project,
unilaterally appoint new Members from the Project community to replace any
unreachable Members and/or to increase the Committee composition to the
required Minimum.

\item \textbf{An Elected Oversight Committee.} The \signatories{}, each a
  hereto, shall initially [FIXME: form or comprise] the \leadershipbody{} as a
  Project Committee (``Committee'') to represent the Project in its official
  communication with NumFOCUS.  The Committee shall hereafter be elected by
  community members of the Project as designated by the Committee or a
  subcommittee of the Committee (the ``Community Members'').

The positions on the Committee will be on a two-year staggered basis ([FIXME:
  some portion] of the initial board seats will be for one year).  The members
of the Committee may be removed from the position at any time by a majority
vote of the Community Members.  Upon the resignation or removal of a member of
the Oversight Board, the Community Members shall elect a replacement Community
Member to serve on the Committee.

The Committee will elect a single individual to communicate with NumFOCUS (the
``Representative'') and shall notify NumFOCUS promptly following the election
of a new Representative.  The Representative will have the authority to
instruct NumFOCUS on the Project's behalf on all matters.

This section may be modified by a vote of at least $\frac{3}{4}$ths of the
Community Members, with the consent of NumFOCUS, such consent not to be
unreasonably withheld.


\end{itemize}

Note again that the above are merely examples, not a list of options.
NumFOCUS' goal is to draft the Representation section to match the existing and
natural leadership structure of the Project, so each project usually has a
uniquely worded Representation section. ]
